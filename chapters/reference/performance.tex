
%|  Name  | TODO | ONGOING | DONE |
%|--------|------|---------|------|
%| Dana   | x    |         |      |
%| Gerd   | x    |         |      |
%| Glenn  | x    |         |      |
%| Jordan | x    |         |      |
%| Luke   | x    |         |      |
%| Matt   | x    |         |      |
%| Neil   | x    |         |      |
%| Scot   | x    |         |      |


\todo[inline]{Owner: Neil -- Priority: High -- Effort: M -- Completion: 50\%}

\paragraph{Scope.}

\begin{itemize}
    \item Library dos and dont's
    \item Skip list \& hash table use

The skip list is a simple in-memory data structure that allows O(log N) key/value insertion, lookup, and removal. Skip lists are used very widely in the library for many different tasks including property list operations and I/O chunk selection (among many others). While skip lists should theoretically be efficient for these use cases as long as N does not grow extremely large, profiling has found that skip list operations are a frequent cause of library underperformance, even though the N values (number of properties, number of IDs) are not extremely large. We currently suspect this underperformance is due to the shape of the skip list data structures causing frequent CPU cache misses. More investigation is needed, and it may be possible to improve skip list performance with additional work. It is now recommended to avoid using the skip list in performance critical sections of the code wherever possible. The primary motivation behind the context package (\texttt{H5CX}) is to avoid invoking skip list operations when handling property lists whenever possible. Unlike a classic skip list, HDF5's skip list is deterministic and does not use the random number generator and therefore will not interfere with the application's random numbers.

In order to improve performance, we have more recently incorporated the open source uthash hash table into HDF5, and begun using it in places that previously would use the skip list. A hash table offers, in general, O(1) insertion, lookup, and removal at the cost of higher memory usage. Therefore, a hash table can be expected to scale better than a skip list when handling large numbers of objects (again, at the cost of high memory usage). In practice, we have also found uthash to be more performant even with smaller numbers of objects. We recommend that developers use uthash instead of skip lists whenever possible and whenever memory usage is not critical.

When objects inserted into these structures need to be iterated over in key order, the advantage of uthash is less apparent. In this case, iterms in the skip list are already in the correct order, but the uthash hash table must be sorted with \texttt{HASH\_SORT}, which is an O(log N) operation. It is likely that uthash is still faster in this case, but this needs more investigation.

    \item Allocations and deallocations

In order to minimize the number of calls to system memory management functions, the HDF5 library has implemented the free list (\textt{H5FL}) package. This package serves as a cache of allocated memory blocks and is effectively an intermediate memory manager between the library and the operating system. Regular free lists are the most commonly used type of free list, and serve as a cache of allocations of a single structure type. Other free list types are sequence, array, and factory free lists athat are used for variable sized allocations and are less frequently used, and may not be as efficient as simply using the system memory manager. More investigation is needed.

We have found mixed results when evaluating the performance of the free lists in HDF5. There is a configure option to disable free lists, and sometimes we see greater performance with free lists disabled. It is likely that, when performing \texttt{calloc} operations, free lists are less efficient than simply using the system \texttt{calloc()}, since the free list must \texttt{memset} the buffer, while the system can return a block from a page it knows has already been set to zero. For this reason, we recommend avoiding the use of \texttt{H5FK\_CALLOC()} whenever possible, and when that is not possible, consider using the system \texttt{calloc()} instead, unless there is reason to believe the free list will be faster. Note that, since you cannot mix free list and system memory management calls on the same memory block, if \texttt{calloc} is only needed some of the time it might require the use of profiling to determine which method is best.

    \item Parallel performance considerations
\end{itemize}