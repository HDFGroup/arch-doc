
%|  Name  | TODO | ONGOING | DONE |
%|--------|------|---------|------|
%| Dana   | x    |         |      |
%| Gerd   | x    |         |      |
%| Glenn  |      |   x     |      |
%| Jordan | x    |         |      |
%| Luke   | x    |         |      |
%| Matt   | x    |         |      |
%| Neil   | x    |         |      |
%| Scot   | x    |         |   x   |


% \subsection{VOL Connectors}

% \subsection{Native VOL Extension Interfaces}


% \subsubsection{User-Defined Datatypes \& Conversions}

% \subsubsection{User-Defined Identifiers}

% \subsubsection{User-Defined Links}

% \subsubsection{User-Defined Properties}

% \subsubsection{User-Defined Error Classes}

% It is possible to define custom error classes to handle specific types of errors or exceptions that may occur internally in HDF5. Custom error classes are used when implementing extensions to HDF5, such as VOLs, Filters, and VFDs. Developers can provide detailed error reports for their specific extensions, improving debugging capabilities for both developers and end-users.

% Any more use cases of custom error classes?
% Example of registering the error stack: H5Eregister_class
% Example of how custom error stack is applied in a limited extension

% \subsubsection{Filters}

% \subsubsection{Virtual File Drivers (VFD)}

\subsection{Language Bindings\label{subsec:LangBindings}}

\subsubsection{General Considerations}
Language-binding interfaces serve the purpose of providing access to the C APIs and parameters to other programming languages. HDF5 offers three language wrappers for the C APIs: Fortran (Section \ref{subsec:Fortran}), Java (Section \ref{subsec:Java}), and C++. Each language component is developed separately from the C library component, and the build systems play a crucial role in obtaining information from the C library to ensure interoperability between the language interfaces.

The interface library aims to incorporate native language elements into the targeted language's API whenever possible. For example, the Fortran APIs avoid using \texttt{ISO\_C\_BINDING} KIND types such as \texttt{C\_INT} and \texttt{C\_FLOAT} in the API interfaces because most Fortran applications don't declare INTEGERs and REALs in the context of C interoperability. Instead, the Fortran wrappers handle the C interoperability at the wrapper level, making it unnecessary for a Fortran programmer to be familiar with C types in order to use the Fortran APIs.

The language bindings maintain their source, tests, and examples in isolation from the main C library tree. They can be enabled and disabled at build time without affecting the main C library. The testing schema for the language bindings has a different focus than the C library tests. Testing the language bindings focuses on their functionality and interoperability with the C APIs. In contrast, C testing aims to ensure the functionality of the C APIs in terms of the HDF5 library as a whole. Hence, the API wrapper testing, for the most part, assumes that the C API functionality has already been tested.

\subsubsection{Fortran\label{subsec:Fortran}}

The Fortran wrappers can be found in \textit{fortran/src} in the HDF5 source code. The HDF5 modules topics are mapped to their corresponding
\emph{H5{[}}\textbf{\textsl{A}},\textbf{\textsl{D}},\textbf{\textsl{S}},\textbf{\textsl{T}},\textbf{\textsl{E}},\textbf{\textsl{ES}},\textbf{\textsl{F}},\textbf{\textsl{Z}},\textbf{\textsl{G}},\textbf{\textsl{I}},\textbf{\textsl{L}},\textbf{\textsl{O}},\textbf{\textsl{P}},\textbf{\textsl{R}},\textbf{\textsl{VL}}\emph{{]}ff.F90} files, where \textbf{\textsl{A}}-- Attribute, \textbf{\textsl{D}} -- Dataset, \textbf{\textsl{S}} -- Dataspace, \textbf{\textsl{T}}
-- Datatype, \textbf{\textsl{E}} -- Error handing, \textbf{\textsl{ES}} -- Event Set, \textbf{\textsl{F}} -- Files, \textbf{\textsl{Z}} -- Filters, \textbf{\textsl{G}} -- Groups, \textbf{\textsl{I}} -- Identifiers,  $\varnothing$ -- Library General, \textbf{\textsl{L}} --Links, \textbf{\textsl{O}} -- Objects, \textbf{\textsl{P}} -- Property
List, \textbf{\textsl{R}} -- References, \textbf{\textsl{VL}} --VOL connector. The older Fortran wrappers currently call a C wrapper
to call the C APIs from. The C wrappers are located in \emph{H5}\textbf{\textsl{{[}A}},\textbf{\textsl{D}},\textbf{\textsl{S}},\textbf{\textsl{T}},\textbf{\textsl{E}},\textbf{\textsl{ES}},\textbf{\textsl{F}},\textbf{\textsl{Z}},\textbf{\textsl{G}},\textbf{\textsl{I}},\textbf{\textsl{L}},\textbf{\textsl{O}},\textbf{\textsl{P}},\textbf{\textsl{R}},\textbf{\textsl{VL}}\emph{{]}f.c}.
Recent wrapper development, however, forgo calling intermediate C stubs and calls the C APIs directly from Fortran. The idea is to eliminate the necessity of the intermediate C wrappers, eventually using Fortran 2003 (F2003) and Fortran 2008 (F2008) features.

The F2003 and F2008 standards have simplified the interoperability between the C and the Fortran wrappers to a great extent. The Fortran wrappers take full advantage of this feature, which has improved their portability with the main C HDF5 library, reduced development time invested in Fortran, and made it easier to maintain them. These improvements were incorporated into the HDF5 library, starting with the 1.10 release.
The following features of F2003 and F2008 are commonly used, and their availability is required for the wrappers:
\begin{itemize}
    \item The \texttt{ISO\_C\_BINDING} and \texttt{ISO\_FORTRAN\_ENV} modules are used.
    \item C pointers are represented by an opaque derived type called \texttt{C\_PTR}, and \texttt{C\_FUNPTR} is used for callback functions.
    \item All Fortran APIs use the \textit{BIND(C, name=\textbf{C API})} convention, where \textit{\textbf{C API}} matches the name of the C API.
\end{itemize}
With these new features, most, if not all, of the C wrappers can be eliminated by calling the HDF5 C APIs directly from the Fortran wrappers.

Lastly, the Fortran wrappers avoid modifying a user's buffer and try to minimize memory copies and allocations whenever possible. Consequently, when a C application reads data written from a Fortran program, the data appears to be transposed due to the difference in the storage order between C (row-major) and Fortran (column-major). For instance, if Fortran writes a 4x6 two-dimensional dataset to a file, a C program will read it into memory as a 6x4 two-dimensional dataset. It is worth noting that the HDF5 C utilities \textit{h5dump} and \textit{h5ls} display transposed data when data is written from a Fortran program.

An overview of the files, file generation, and other Fortran wrapper development practices are updated and detailed in \textit{README.md}, located in \textit{fortran/src}.

\subsubsection{Java\label{subsec:Java}}

The java wrappers can be found in \texttt{java/src} in the HDF5 source code. These wrappers
are separated into two main components: the Java wrapper functions and the
\href{https://docs.oracle.com/en/java/javase/17/docs/specs/jni/index.html}{Java Native Interface}
(JNI) C functions.

The Java wrapper functions primarily consist of simple function declarations in \\
\texttt{java/src/hdf/hdf5lib/H5.java} which contain the `native` keyword, informing the Java
Virtual Machine that those functions will be implemented by a matching JNI C function.
Additional supporting Java infrastructure can be found in the following files/directories:

\begin{itemize}
  \item \texttt{java/src/hdf/hdf5lib/H5.java} \\
    In addition to containing the Java wrapper function declarations, this file also contains
    code (in \texttt{loadH5Lib}) to load the C library that contains all of the JNI C functions
    that implement the Java wrapper functions. This code is run when the \texttt{H5} class is
    initialized and must successfully find and load the C library for the Java wrappers to
    operate correctly.
  \item \texttt{java/src/hdf/hdf5lib/HDF5Constants.java} \\
    This file defines a Java class that contains public variables for all of the constants that
    are available in the public C interface, usually as a \#define macro, such as
    \texttt{H5F\_LIBVER\_LATEST}. Each variable has a name matching its corresponding macro in
    the C library, and the value for each variable is populated by having a similarly named
    JNI C function that simply returns the relevant value.
  \item \texttt{java/src/hdf/hdf5lib/HDFArray.java} \\
    This file defines functions to convert multi-dimensional arrays of bytes into
    arrays of Java objects and vice versa. Its main goal is to facilitate and simplify the
    I/O pathway between Java and C.
  \item \texttt{java/src/hdf/hdf5lib/HDFNativeData.java} \\
    This file defines Java wrapper functions for methods that support \texttt{HDFArray.java}
    during the data element conversion process.
  \item \texttt{java/src/hdf/hdf5lib/callbacks} \\
    This directory contains source files that define Java interfaces, functions, etc., necessary
    for being able to make a call from Java to an HDF5 API that accepts a pointer to a callback
    function, such as \texttt{H5Aiterate2} (which accepts an \texttt{H5A\_operator2\_t}).
  \item \texttt{java/src/hdf/hdf5lib/exceptions} \\
    This directory contains source files defining Java classes representing errors that
    can be returned from the HDF5 Java wrappers. The main error class defined is the \texttt{HDF5Exception}
    class, which extends Java's \texttt{RuntimeException}. Two other
    error classes are derived from this class, \texttt{HDF5LibraryException} and \texttt{HDF5JavaException}, from which several
    other error classes are derived. Errors derived from the \texttt{HDF5LibraryException} class
    correspond to errors that can be returned from the HDF5 API, while errors derived from the
    \texttt{HDF5JavaException} class corresponds to errors that occur in the Java wrapper functions
    and supporting infrastructure or in the JNI C code that implements the Java wrapper functions.
  \item \texttt{java/src/hdf/hdf5lib/structs} \\
    This directory contains source files that define Java classes which represent the C structures
    passed to certain HDF5 API calls, such as the \texttt{H5O\_info\_t} structure for \texttt{H5Oget\_info}.
\end{itemize}

The Java wrapper functions currently do not handle any versioning of the HDF5 APIs and instead
are typically updated to reflect the latest version of a versioned HDF5 API. For example, as of
the time of writing the Java wrapper for \texttt{H5Oget\_info} is simply called \texttt{H5Oget\_info},
but its parameters and the struct it returns (\texttt{H5O\_info\_t}) reflect the \texttt{H5Oget\_info3}
C API.

The JNI C functions reside in the source files under \texttt{java/src/jni}. In order for the
Java Virtual Machine to be able to match a Java wrapper function with its C implementation, the
JNI uses a name-mangling scheme somewhat similar to C++. The name for a function must follow
the convention documented at \href{https://docs.oracle.com/en/java/javase/17/docs/specs/jni/design.html#compiling-loading-and-linking-native-methods}{Compiling, Loading and Linking Native Methods},
where the name of the function is prepended with a prefix of "Java\_", the fully-qualified Java method
name (interspersed with underscores) and any needed escape sequences. For example, the JNI name for
the \texttt{H5Dread} function is as follows:

\begin{verbatim}
Java_hdf_hdf5lib_H5_H5Dread
\end{verbatim}

To conform to the specification that the JNI expects, each JNI C function should be marked with
\texttt{JNIEXPORT} and \texttt{JNICALL} and must contain \texttt{JNIEnv *env, jclass clss} as the
first two parameters, followed by the JNI equivalents for each of the parameters to the matching
Java method. The parameters must also come in the same order as those in the matching Java method.
For example, the full signature for the \texttt{H5Dread} function is:

\begin{verbatim}
JNIEXPORT jint JNICALL
Java_hdf_hdf5lib_H5_H5Dread(JNIEnv *env, jclass clss,
                            jlong dataset_id, jlong mem_type_id,
                            jlong mem_space_id, jlong file_space_id,
                            jlong xfer_plist_id, jbyteArray buf,
                            jboolean isCriticalPinning)
\end{verbatim}

\texttt{JNIEnv *env} is a special parameter passed to each JNI C function by the JNI itself and
is used to call C APIs that allow direct interaction with Java objects. \texttt{jclass clss} is
also a special parameter passed to each JNI C function by the JNI but is mostly unused. For the
other parameters, one can refer to \href{https://docs.oracle.com/en/java/javase/17/docs/specs/jni/types.html}{JNI Types and Data Structures}
when translating between Java types and JNI types.

HDF5 JNI C functions are typically implemented as very thin wrappers around the HDF5 C API. Usually,
simple argument checking is performed upon function entry, any necessary argument pinning may occur
(described below), the relevant HDF5 C API call is made with the parameters passed to the wrapper
function and then some cleanup is performed before returning from the function. Error handling in
HDF5 JNI C functions are very similar in design to the HDF5 C library. Each function has a 'done:'
label, and a set of macros is used to jump to this label with \texttt{goto} when an error occurs.
Once an error occurs, these macros will construct an instance of the corresponding HDF5 Java exception
class mentioned above and return that exception to the Java interface so it can be handled appropriately. If not handled at the Java interface level, the Java Virtual Machine may crash.

Due to the fairly large overhead incurred when crossing the boundary between Java and C code, there
are a few best practices one should follow when designing new HDF5 JNI C functions:

\begin{itemize}
  \item HDF5 Java wrapper functions should generally be designed such that the number of transitions
  between Java and C code to accomplish a task is minimized and as much work as possible is
  done within the C code portion of the wrapper function.
  \item To reduce the overhead of accessing Java array objects within C code, the JNI provides a
  process typically referred to as "pinning". This allows one to "pin" an array and get a direct
  pointer to the elements of the array, rather than iterating through the array and accessing the
  elements individually with JNI calls. The \texttt{h5jni.h} header provides macros that wrap around
  the relevant JNI calls to pin arrays of each primitive datatype, such as \texttt{PIN\_INT\_ARRAY},
  \texttt{PIN\_INT\_ARRAY\_CRITICAL} and \texttt{UNPIN\_INT\_ARRAY}. These macros will return a pointer
  to the pinned array, as well as a boolean value that informs the caller whether the JNI had to make
  a copy of the array in order to pin it properly. If a copy had to be made, the caller \textbf{must}
  be sure to call the matching "unpin" macro (e.g., \texttt{UNPIN\_INT\_ARRAY}) before returning from
  the wrapper function. Otherwise, any changes to the array will not be reflected, and any resources
  used by the copied array will not be released. If performance is of the utmost importance in a wrapper
  function that deals with array objects, one may use the "critical" versions of these pinning macros
  rather than the regular versions. This approach will more likely give the caller an uncopied
  version of the array in question. Still, there are restrictions around when and how these particular
  macros can be used that the caller should be aware of (see \href{https://docs.oracle.com/en/java/javase/17/docs/specs/jni/functions.html#getprimitivearraycritical-releaseprimitivearraycritical}{GetPrimitiveArrayCritical, ReleasePrimitiveArrayCritical}).
  Further information around this pinning process can be found at \href{https://docs.oracle.com/en/java/javase/17/docs/specs/jni/design.html#accessing-primitive-arrays}{Accessing Primitive Arrays}
  and \href{https://docs.oracle.com/en/java/javase/17/docs/specs/jni/functions.html#getprimitivetypearrayelements-routines}{Get$<$PrimitiveType$>$ArrayElements Routines}.
\end{itemize}
