%\todo[inline]{Owner: Dana -- Priority: High -- Effort: M -- Completion: 90\%}

\section{Operating Systems}

\subsection{Windows}
\begin{itemize}
    \item SWMR should work on Windows with local file systems (NOT SMB shares), though testing is not a robust as on MacOS/Linux
    \item Parallel HDF5 is poorly tested on Windows
    \item The direct VFD is not supported as it uses the POSIX \texttt{O\_DIRECT} flag
    \item Thread-safety uses win32 threads
    \item Some API calls use \texttt{off\_t}, which is a 32-bit type on Windows (we use \texttt{\_\_int64} internally)
    \item Unicode file paths are not well-supported
\end{itemize}

\section{Build Systems}

All features should be supported in both the Autotools and CMake.

\section{Thread-Safety}

Thread-safety is currently implemented via a global API lock, which works with any API call in the HDF5 C library. This includes parallel HDF5, even though the build system will disallow this combination in earlier versions of the library (for no particular reason other than we didn't test it).

The global lock is not implemented in any of the wrapper or higher-level libraries, so they are NOT thread-safe:

\begin{itemize}
    \item High-Level Library
    \item C++ API
    \item Fortran API
    \item Java API
\end{itemize}

\newpage

\section{Feature Compatibility}

\begin{minipage}{\textwidth}
    \begin{center}
        \begin{tabular}{|| c | c | c | c ||}
            \hline
            \textbf{Feature} & \textbf{Serial} & \textbf{SWMR (serial)} & \textbf{Parallel} \\
            \hline
            Asynchronous Operations & 1 & 1 & 1 \\
            \hline
            Compression (Filtering) & Y & Y & 11 \\
            \hline
            Direct Chunk I/O & Y & Y & 8 (?) \\
            \hline
            Direct VFD & 2 & 2 & 2 \\
            \hline
            External Data Storage & Y & Y & 9 \\
            \hline
            External Links & Y & 7 & Y \\
            \hline
            Evict-on-Close & Y & Y & N \\
            \hline
            File Image & Y & N & N \\
            \hline
            Free-Space Tracking & Y & N & Y \\
            \hline
            Metadata Cache Image & Y & Y & Y \\
            \hline
            Metadata Page Buffering & Y & N & N \\
            \hline
            Metadata Updates & Y & 5 & 6 \\
            \hline
            Multi-Dataset I/O & 3 & 3 & 3 \\
            \hline
            Selection I/O & 3 & 3 & 3 \\
            \hline
            User-Defined Links & Y & 7 & Y \\
            \hline
            Variable-Length Datatypes & Y & N & 7 \\
            \hline
            Vector I/O & 3 & 3 & 3 \\
            \hline
            Virtual Dataset (VDS) Layout & 4 & 4 & 10 \\
            \hline
        \end{tabular}

        \footnotesize
        \begin{enumerate}
            \item Requires a suitable VOL connector (NOT the native connector)
            \item Requires the POSIX \texttt{open(2)} call to accept the \texttt{O\_DIRECT} flag
            \item Requires a suitable virtual file driver (e.g., MPI-I/O) to see a benefit
            \item Does not support point selections
            \item Extending datasets only
            \item Requires collective metadata operations if writing to a common file
            \item Read-only
            \item Not with filtered chunks
            \item Independent only
            \item Parallel VDS has limited functionality
            \item Limited to early allocation
        \end{enumerate}
        \normalsize

    \end{center}
\end{minipage}

\newpage

\paragraph{VFD stackability and features}
\begin{center}
    \begin{tabular}{|| c | c ||}
      \hline
      \rotatebox{90}{Pass-Through} & family, mirror, multi/split, onion, splitter, subfiling \\
      \hline
      \rotatebox{90}{Terminal} & core, direct, HDFS, Hermes, log, MPI-I/O, sec2, ros3, stdio\\
      \hline
    \end{tabular}

    \footnotesize
    \textbf{NOTE:} The onion and subfiling VFDs are currently hard-coded to use the sec2 VFD.
    \normalsize
\end{center}

%\paragraph{VOL connector stackability}