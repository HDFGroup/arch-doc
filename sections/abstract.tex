
%|  Name  | TODO | ONGOING | DONE |
%|--------|------|---------|------|
%| Dana   | x    |         |      |
%| Gerd   | x    |         |      |
%| Glenn  | x    |         |      |
%| Jordan | x    |         |      |
%| Luke   | x    |         |      |
%| Matt   |      |         | x    |
%| Neil   | x    |         |      |
%| Scot   |     |         |  x    |

\documentclass[../main.tex]{subfiles}
 
\begin{document}
\begin{abstract}
\begin{quote}
    I don't mean a bricks and mortar architect. I mean an architect as used in the words \textit{architect of foreign policy.} I mean architect as in the creating of systemic, structural, and orderly principles to make something work -- the thoughtful making of either artifact, or idea, or policy that informs because it is clear. I use the word information in its truest sense. Most of the word information contains the word \textit{inform,} so I call things information only if they inform me, not if they are just collections of data, of stuff. (Saul Wurman)
\end{quote}
This document aims to provide a thorough understanding of the inner workings of the HDF5 library by delving into its underlying principles. It covers the systemic, structural, and orderly aspects that make the library function in a clear and informative manner. By going through this document, one can gain insights into the library's architecture and how to use it efficiently. Additionally, it will provide an overview of the various techniques available to simplify the understanding of the operations of the HDF5 library.
\end{abstract}
\vspace{8cm}
\textbf{Acknowledgment}

This work was supported by the Exascale Computing Project (17-SC-20-SC), a joint project of the U.S. Department of Energy’s Office of Science and National Nuclear Security Administration, responsible for delivering a capable exascale ecosystem, including software, applications, and hardware technology, to support the nation’s exascale computing imperative.

\end{document}